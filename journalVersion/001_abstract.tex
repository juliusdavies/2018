
% MWG's 2nd attempt at an abstract (Jan 27, 2011):
Deployed software systems are typically composed of many pieces, not all of
which may have been created by the main development team.  Often, the
provenance of included components --- such as external libraries or cloned
source code --- is not clearly stated, and this uncertainty can introduce
technical and ethical concerns that make it difficult for system
owners and other stakeholders to manage their software assets.
In this work, we motivate
the need for the recovery of the provenance of software entities
by a broad
set of techniques that could include signature matching, source code fact
extraction, software clone detection, call flow graph matching, string
matching, historical analyses, and other techniques.
We liken our provenance goals to
that of Bertillonage, a simple and approximate forensic analysis technique
based on bio-metrics that was developed in $19^{th}$ century France before
the advent of fingerprints.  As an example, we have developed a fast,
simple, and approximate technique called \emph{anchored signature matching}
for identifying the source origin of binary libraries within a given Java
application.  This technique involves a type of structured signature
matching performed against a database of candidates drawn from the Maven2
repository, a 275GB collection of open source Java libraries.
To show the approach is both valid and effective, we conduct
an empirical study on 945 jars from the Debian GNU/Linux distribution,
as well as an industrial case study on 81 jars from an e-commerce
application. 



% Kent Beck's 4-sentence abstract algorithm:
%
% http://plg.uwaterloo.ca/~migod/research/beckOOPSLA.html
%
%
% 1. The first states the problem.
% 2. The second states why the problem is a problem.
% 3. The third is my startling sentence.
% 4. The fourth states the implication of my startling sentence. 

% JWD's attempt at an abstract:

% 1. The first states the problem.
%Deployed software systems are typically composed of many pieces, not all of
%which may have been created by the main development team.  Often, the
%provenance of included components --- which may include external libraries
%or cloned source code --- is not clearly stated.

% 2. The second states why the problem is a problem.
%This makes it difficult for developers to
%manage and maintain their software assets.  Unfortunately, 
%existing techniques for discovering provenance, while exact, tend to be expensive
%and of limited use.

% 3. The third is my startling sentence.
%In this work we propose a new framework for provenance:
%software Bertillonage.  Our framework takes as inspiration a
%French forensic technique developed in the 19th century,
%before the advent of fingerprints.

% 4. The fourth states the implication of my startling sentence. 
%A case study of a proprietary e-commerce Java application
%tested the feasibility of our Bertillonage approach.  By matching class
%signatures against signatures in the Maven2 central repository --- a 150GB 
%collection of open source Java libraries --- we disinterred valuable provenance information.
%Our technique, while approximate, proved itself to be simple and surprisingly fruitful.


% old abstract:
%
%Deployed software systems are typically composed of many pieces, not all of
%which may have been created by the main development team.  Often, the
%provenance of included components --- which may include external libraries
%or cloned source code --- is not clearly stated.  This raises a number of
%both technical and ethical/legal concerns.  Technically, it is often hard
%to maintain such a system if its external dependencies are not well
%documented.  Ethically, code fragments that have been copied from other
%sources, such as open source software, may not have licences that are
%compatible with the released system.  In this work, we motivate the need
%for recovery of the provenance of software entities by a broad set of
%techniques that include source code fact extraction, software clone
%detection, call flow graph matching, string matching, and historical
%analyses.  We liken our goals to that of Bertillonage, a simple and
%approximate forensic analysis technique based on bio-metrics that was
%developed in France before the advent of fingerprints.
%
%As a motivating example of this kind of work, we consider the PCI DSS
%security standard for e-commerce, which requires that an application should
%provide precise version information about any libraries that are packaged
%with it.  In practice, this information is often not provided and so we
%have sought ways to infer it from available evidence.
%
%We used a single Bertillonage metric of own invention, anchored signature matching,
%to analyze Java libraries from a proprietary e-commerce Java application.
%The application of this single metric allowed us to automatically provide exact
%version information for over 57\% of our sample set, and to narrow the search
%space significantly for another 39\%, providing actionable information on
%96\% of the libraries within the e-commerce application.
%

%%% Local Variables: 
%%% mode: latex
%%% TeX-master: "000_main"
%%% End: 
