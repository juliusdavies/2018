

%   We tried running our analysis against Google's android.jar, a jar not currently contained
%   in the Maven2 repository). Our method decided
%   it must come from Sun/Oracle's jdk sources (jdk-6u20-ea-src-b02-jrl-01\_apr\_2010.zip)
%   with 221 matches.
%   Together with the cayenne-1.2.2.jar results in Table \ref{tab:falseMatch},
%   this shows how our method becomes confused by large clone imports,
%   since these cause a spike in secondary matches, potentially eclipsing the primary
%   matches.  In addition, there is no way for our method to reliably distinguish
%   between primary and secondary matches.  For all our method knows, cayenne-1.2.2 is
%   the canonical location for the commons-collections-3.1 classes, and not the other
%   way around.

%   But this confusion also has its uses:  Apache HttpClient developers recently
%   wanted to know what version of their libraries Google had statically imported
%   into the android.jar file.  By querying android.jar in our index we were
%   able to provide specific version information concerning two libraries published
%   by the Apache HttpClient team:  httpcore-4.0-beta2.jar, and httpclient-4.0-beta1.jar.
%   See Table \ref{tab:android} to see how these were identified.  The main HttpClient
%   developer asked if we could also test against httpclient-4\_0\_API\_FREEZE.jar,
%   an uncommon version he suspected Google had actually imported.  We loaded
%   httpclient-4\_0\_API\_FREEZE.jar into our index and re-ran the query with android.jar.
%   The match-count was the same, but the earlier zip-date caused our method to pick this,
%   certainly not contradicting the main developer's suspicion.  In this way our tool
%   serves as a useful clone detection engine, especially for static imports.


% ------Ties and License Resolution---------
%   One interesting
%   possibility arises when considering ties in conjunction with license requirements:  when an integrator
%   notices they have a license incompatibility, ties may offer potential avenues for resolution.
%   For example, suppose an integrator is initially considering the ApacheV1 commons-codec-1.1.jar, but
%   then realizes they need a GPL-compatible license.  In this case a tie with an ApacheV2 commons-codec-1.2.jar
%   might suggest a good upgrade path for the integrator to satisfy their licensing requirements,
%   since the tie implies the more suitably licensed commons-codec-1.2 is also a reverse-compatible library,
%   thanks to its matching signatures.



%\begin{table*}
%  \centering
%  \caption{Discussion - What version of Apache httpclient is inside android.jar?}
%  \begin{tabular}[htbp]{llr|llr}
%\textbf{Matched binaries}        & \textbf{Zip Timestamp}  & \textbf{Matches}  & \textbf{Matched binaries}             & \textbf{Zip Timestamp}  & \textbf{Matches}  \\
%\hline
%\hline
% httpcore-4.1-alpha1.jar         & 2009-Sep-06  &  84  &  httpclient-4.1-alpha1.jar       & 2009-Dec-10  & 110  \\
% httpcore-4.0.1.jar              & 2009-Jun-19  &  87  &   httpclient-4.0.1.jar            & 2009-Dec-10   & 117  \\
% httpcore-4.0.jar                & 2009-Feb-20  &  87  &   httpclient-4.0.jar              & 2009-Aug-07  & 119  \\
% httpcore-4.0-beta3.jar          & 2008-Oct-15   &  88  &   httpclient-4.0-beta2.jar        & 2008-Dec-18  & 125  \\
% \textbf{httpcore-4.0-beta2.jar} & \textbf{2008-Jun-18} &  \textbf{92}  &  httpclient-4.0-beta1.jar        & 2008-Aug-23  & 127  \\
% httpcore-4.0-beta1.jar          & 2008-Jan-21  &  88   &  \textbf{httpclient-4\_0\_API\_FREEZE.jar}   & \textbf{2008-Jul-24}  & \textbf{127}  \\
% httpcore-4.0-alpha6.jar         & 2007-Oct-05  &  65   &  httpclient-4.0-alpha2.jar       & 2007-Nov-06   &  53  \\
% httpcore-4.0-alpha5.jar         & 2007-Jun-28   &  35  &  httpclient-4.0-alpha3.jar       & 2008-Feb-21   &  67  \\
% jakarta-httpcore-4.0-alpha4.jar & 2007-Apr-23   &  31  &  httpclient-4.0-alpha4.jar       & 2008-May-05  &  83  \\
% jakarta-httpcore-4.0-alpha3.jar & 2006-Dec-07   &  26  & httpclient-4.0-alpha1.jar       & 2007-Jul-15  &  31  \\
% jakarta-httpcore-4.0-alpha2.jar & 2006-Jun-08  &  23   & & & \\
% jakarta-httpcore-4.0-alpha1.jar & 2006-Apr-16   &  21  & & & \\
%\hline
%  \end{tabular}
%  \label{tab:android}
%\end{table*}



\subsection{Threats to Validity{\label{sec:threats}}}

This section discusses the main threats to validity that can affect the
studies we performed. 

In particular, threats to {\em construct validity} may concern imprecision
in the measurements we performed. Our logic for detecting Java and class
files in the Maven2 repository relied on accurate detection of
\mytt{.java} and \mytt{.class} files, as well as \mytt{.jar},
\mytt{.ear}, \mytt{.war}, 
\mytt{.zip}, \mytt{.tar.gz}, \mytt{.tar.bz2}, and \mytt{.tgz}
archives.  No other search patterns were employed, and thus some archives
may have been missed.  This threat is diminished thanks to the very large
amount of data we managed to extract from just those nine search patterns.

%Our subsequent logic for extracting the class signatures could be faulty,
%in particular our Java source parser.  We are less concerned about faults
%in our bytecode analysis, since the \mytt{bce-5.2.jar} tool we used is 4
%years old, very popular, and very well tested.  Bearing in mind that our
%Java source parser is potentially a problem, we believe our results
%nonetheless resemble exactly the shape one would expect for a
%class-signature-index approach, with matches resembling a bell curve that
%drops off as version-numbers diverge from the exact match.  In addition,
%queries involving only bytecode (e.g., queries for bytecode using bytecode)
%resulted in a similar bell curve, alleviating concerns over our source
%parser. We also do not address malicious or intentionally deceptive
%provenance issues.

Threats to {\em internal validity} arise primarily from our technique for
verifying a correct match: we visually check the version number in the
names of jars and zip files.  To address this threat, we samples 945 jars
with known provenance information from Debian, and we also conducted a
thorough byte-by-byte comparison of all our jars.
%We only conducted a thorough byte-by-byte
%comparison in cases where the initial visual check failed.
%RQ3 in particular
%makes the threat clear: are version numbers in jar files at all
%reliable?  
One threat to internal validity is that we rely on authoritative file-names
instead of other information like tags found in version control systems
(VCSs).
%According to current
%tradition in software development the ultimate authority on version is the tag in the version control
%system (VCS).  
%We did not try to address this weakness empirically by comparing tagged VCS code
%against published jars and zips, and such a study would be highly valued here.
%We showed that even in the Maven2 repositories there were
%mislabelings,  but to a lesser extent than the case-study projects.
We hypothesize that developers involved in the creation and/or packaging of
open source libraries for Debian and for the Maven2 repository strive to
publish correct version information, since dependency management systems
rely on such information.


Threats to {\em external validity} concern the generalization of our results.
%We provide a single case study involving a proprietary enterprise application,
%and us such, our study shows feasibility rather than generalizability.
Our sample of 945 Debian jars attempts to minimize this threat, but the
Debian collection may be atypical, for example, most Java developers choose
to develop and deploy applications to the Windows platform.  Could the
Debian sample be missing jars that are more popular on Windows?  We believe
the large size of our Debian sample mitigates this threat.  We postulate
there is a strong tradition of platform-independent development within the
Java community.  Such a tradition, if it exists, would further lessen the
risk of any significant body of Windows-specific or Mac-specific Java
archives being missed by our sample.  Another threat to our external
validity comes from Maven's own composition:  is Maven's repository a good
sample of open source software in the Java eco-system?  Given its critical
position in industry with respect to Java dependency resolution (even
unrelated dependency resolvers such as Ivy use the Maven2 repository), we
believe it is representative.  We have one complaint about its composition:
it contains too many alpha, beta, milestone, and release-candidate
artifacts that are likely of little interest to integrators.
%In future studies we may consider filtering these out.
%We control some of the over-reprentation by enforcing a unique index
%in our database for each Java/class file.
%In this way many popular jars such as \mytt{ow2-util-scan-impl-1.0.18.jar} appear only once in our database,
%despite 113 occurences scattered throughout the Maven2 repository.

\section{Discussion{\label{sec:discussion}}}

%[cite the OOPSLA-2009 Onward! harvard paper....]

What is provenance?   Is \emph{name} and \emph{release number} alone a
suitable representation of provenance for our purposes?  Suppose a given
jar is authoritatively known to be named \emph{foo} and to be release
\emph{x.y.z}.  Our method assigned the highest similarity score to this
single candidate, \emph{foo-x.y.z.jar}, for over $60\%$ of the subject jars
in our case study.  But can provenance really be boiled down to a small
sequence of characters, hyphens, digits and dots.  Does \emph{foo-1.2.3}
constitute provenance?  This question is important, since our technique
assumes it.

Fortunately, for the majority of the jars in this study, and perhaps for
the majority in ``current circulation'' among Java developers, this notion
of provenance is sufficient.  As a thought experiement, imagine asking
random undergraduate students enrolled in \emph{Introduction to Computer
Programming} at any university to download the \mytt{oro-2.0.8.jar} Java
library.  In all likelihood the vast majority would download the same
artifact, even those completely unfamiliar with Java.  Java developers
often manage to avoid name and version collisions among their reusable
libraries.

However, for some jars, this notion of provenance is insufficient.  The
underlying assumption with respect to \emph{name} and \emph{release number}
is that the combination of these two attributes will always result in a
distinct set of software code, an \emph{authoritative} snapshot, frozen in
time.  Among the $81$ jars studied, we observed three challenges to a
\emph{foo-1.2.3} notion of provenance:

\begin{enumerate}

\item Jars that, during their build process, copy classes from other jars.
    For example, \mytt{vreports.jar} contains copies of classes from
    \mytt{itext.jar}.

\item Jars with historically unstable provenance, perhaps due to corporate
    acquisitions, or even internal restructurings within a company.  The
    Sun/Oracle jar named \mytt{xsdlib.jar} is an example of this.  Various
    project websites provide conflicting testimony regarding the jar's
    origins.
%\begin{itemize}
%\item Java Architecture for XML Binding (JAXB)
%\item Java API for XML-Based RPC (JAX-RPC)
%\item Java API for XML Registries (JAXR)
%\item Java Web Services Developer Pack (JWSDP)
%\item Java-ws-xml
%\item Sun Multi-Schema Validators
%\item Project Glassfish
%\end{itemize}
    Each of these projects appears to have taken control of, or at least
    contributed to, \mytt{xsdlib.jar}'s development at some point in its
    history.  The answer may very well be a combination of the projects we
    observed, which each project contributing to different phases of
    \mytt{xsdlib.jar}'s evolution.  In cases such as these, our
    Bertillonage results can resemble a hall of mirrors.  More expensive
    analysis methods, such as sending questions to project mailing lists,
    or analyzing version control repositories are required.

\item Altered jars, e.g., a particular \mytt{foo-1.2.3.jar}, may contain
    $10$ classes, whereas another jar with the same name and release
    information may contain only $9$ classes.  In some cases these $9$ are
    a proper subset of the $10$.  Perhaps a user of the library has
    customized it by adding or removing a class.  Which archive is
    authoritative in this case?  We have examples of this in our data.

\end{enumerate}

In the face of these challenges our Bertillonage approach was surprisingly
fruitful.  Our simple Bertillonage metric could readily accommodate \#1
(emcompassed jars).  Challenges \#2 (unstable provenance) and \#3 (altered
jars) always required additional narrowing work, and yet our approach
nonetheless still revealed when these particular challenges were occurring.
Rather than reinforce our initially narrow notions of provenance, thanks to
the simplicity of our metric, and particularly thanks to an immense (and
messy) data source such as Maven2, our study outlined what future
provenance research must tackle.

\subsection{A Foundation for Higher Analyses}

Developing, deploying, and maintaining software systems can involve many
diverse groups within --- and external to --- an organization.  Each of these
groups may require different knowledge about the software systems they are
involved with.  For example, testers, developers, system administrators,
salespeople, managers, executives, auditors, owners, and other stakeholders
may have specific questions they need answered about an organization's
software assets.  A salesperson may have a technically demanding client
that insists on a specific release of a particular library.  The security
auditor wants to make sure no libraries or copy-pasted code fragments
contain known security holes.  The license auditor wants to know if her
license requirements are being fulfilled.  The manager wants to know how
risky an upgrade to the latest release of a popular object-relational
database mapping library might be.  As noted in
section~\ref{sec:evaluation}, provenance forms a critical foundation upon
which these higher level analyses rely.  Without reliable provenance
information in place these stakeholders cannot even begin to find answers
to their questions.


Provenance information is also important to the software developers
responsible for importing and integrating libraries and code fragments into
their software systems.  Therefore name and release information is often
encoded directly into an artifact's file name (e.g., \mytt{oro-2.0.7.jar}).
But sometimes developers may omit the release numbering, or they may
mistype it.  Also, as we noted earlier, in some cases an artifact
internally encompasses additional artifacts, rendering the file name
inadequate for communicating the versions of the encompassed releases.  For
these reasons, higher level analyses cannot depend on filename alone.

The specific metric we introduced here, anchored signature matching, will
by no means be the final word on software Bertillonage.  But we found our
simple metric to be effective.  For the 945 Debian jars, arguably
an atypically challenging dataset, our approach was able to supply high quality provenance
information for over $75\%$ of the subject archives, including complex
cases where an archive encompassed other archives.  Of course some manual
effort was required in our case study to narrow all matched candidates to
single exact matches,
%\footnote{Some
%high level analyses can tolerate a range of matches, e.g., a software license might
%remain constant across several versions of a library.},
but the original filename was correct for the majority of these, and so the
manual effort was minimal.  Our result minimizes the risk of relying on
filenames exclusively when performing higher level analyses that depend on
provenance.  We also note the excellent binary-to-binary results we
obtained can serve as a bridge to improved binary-to-source results:  with
a single binary match, manually locating the corresponding source archive
(especially in the open source world) is trivial.  This ``bridging'' idea
mitigates the downside of our inferior binary-to-source results.

Our technique also performed well in a separate informal exercise to
determine the moment of a copy-paste of class files.  We noticed the
developers of \mytt{httpclient.jar}, an open source Java library, had posed
a question on their mailing list: when did Google Android developers
copy-paste \mytt{httpclient.jar} classes into
\mytt{android.jar}?\footnote{See email from Bob Lee to dev@hc.apache.org on
18 Mar 2010 23:47:14 GMT, subject "Re: HttpClient in Android."} They wanted
to know this to evaluate how hard it would be for Google to import a more
recent version of their jar.  We employed our technique to answer the
original question on the mailing list, and the main developer confirmed our
result.  We initially identified \emph{4.0-beta1} as the moment of the
copy-paste.  The developer asked if we could also test against
\emph{4\_0\_API\_FREEZE}, an uncommon version he suspected Goo\-gle had
actually imported.  We loaded the \emph{FREEZE} release into our index and
re-ran our analysis.  This resulted in both \emph{4.0-beta1} and
\emph{4\_0\_API\_FREEZE} being returned as equally likely matches for
\mytt{android.jar}.

We were successful in narrowing the search space for the moment of a
copy-paste to just two versions.  In addition, the \mytt{httpclient.jar}
exercise motivated future work. Precedent and subsequent releases diverge
with respect to the cardinality of their intersecting signatures.  Our
anchored signature match is not just useful for finding exact matches.  It
could also prove useful at measuring the distance between versions, which
in turn could be useful for performing risk assessment of releases.


As stated earlier, we performed a license audit and security audit using
the provenance information unearthed from the case study.  The results of
these higher analyses proved useful: the license audit pinpointed a jar
where some versions used the GNU Affero license, while other versions used
LGPL; similarly, the security audit located a jar with a known security
hole.  The organization found the results from both of these audits
valuable, and steps were taken to address both issues in their application.


%%% Local Variables: 
%%% mode: latex
%%% TeX-master: "000_main"
%%% End: 

