
\section{Rough Sketch of Formal Method}




Goal and Definitions:
---------------------

1.  We want to rank any pairing of a binary archive file
    with a source archive file, ie. is a given pair more or less related
    than other pairings?  We ignore file names when determining the rank,
    so the fact a binary and source archive might have very similar
    names (e.g. foo-1.2.3-src.zip vs. foo-1.2.3.jar) does not effect the
    rank.
   
2.  A source archive is defined as any zip, tgz, tar, tar.bz2 or jar file
    that contains Java source files (*.java) at any depth.

3.  A binary archive is defined as any zip, tgz, tar, tar.bz2 or jar file
    that contains Java binary files (*.class) at any depth.

4.  Archive files that contain both *.java and *.class are measured twice:
    once as a source archive, and once as a binary archive.

5.  We assume that when an upstream project releases both foo-1.2.3-src.zip
    and foo-1.2.3.jar there is a 1-to-1 mapping between the *.java and *.class
    files therein.  Unit tests often break this assumption, since they are
    usually included in the source archive but not in the binary archive.


Analyzing An Archive File
---------------------

1.  In the previous section we defined a string representation of source
    and binary Java files that is identical when the source file creates
    the binary file [JWD:  need a better word for this].

2.  Recall:  the string representation is based on attributes of the class
    and its methods which are easily extractable from both the binary and
    source forms.  These attributes might include:

      - Method names, class name.

      - Class inheretance tree.

      - Method parameter types
        (note:  parameter names are not extractable from the binary form,
         but parameter type information is).

      - Method order.  We found no example of a Java compiler re-ordering
        the methods in a class from the order they appear in the original
        source file.

      - Other attributes.

3.  Our string representation over-matches:  there are cases where an
    identical representation is generated even though the source file
    \em{does not} create the binary file.  Consider the following two
    source files. In this example it is impossible to identify the
    differentiating attribute (+ instead of *) without fully decompiling
    the binary forms:

    // Version 1.0
    package one.two.three;
    public class Four {
      public static int getNumber() { return 2 + 2; }
    }

    // Version 2.0
    package one.two.three;
    public class Four {
      public static int getNumber() { return 2 * 2; }
    }

    Nonetheless, since we include package and class name among our extractable
    attributes, matches such as these indicate a relationship (ie. code
    evolution) between the files.  Cases of random co-incindental matches are
    practically impossible; developers on distinct projects explicitly takes
    measures to avoid such naming collisions since linking bugs could result in
    their applications.


Ranking The Pairs
---------------------

1.  A user queries our code-search engine by uploading a binary archive file.
    They are essentially asking, "Find the source archive with the largest
    number of matching string representations."

2.  We generate string representations for each binary file inside the uploaded
    file.  A large archive such as android.jar might generate thousands of
    these 'signatures' or 'fingerprints', whereas a small archive such as
    commons-codec.jar might generate less than twenty:  one for each *.class
    file therein.

3.  We then query our index to find every source archive file that contains at
    least one matching fingerprint.

4.  Of these we evaluate each pairing as a set operation:

    For each pair:
      [binary fingerprints]  INTERSECT  [source fingerprints]

5.  We return the pairing with the largest intersection.
    Note:  For convenience all of the above steps use a SHA1 hash of the
    string representation.


