\startfirstchapter{Introduction}
\label{chapter:introduction}


Deployed software systems often include code drawn from a
variety of sources.  While the bulk of a given software system's
source code may have been developed by a relatively stable set of known
developers, a portion of the shipped product may have come from
external sources.  For example, software systems commonly require the use
of externally developed libraries, which evolve independently from the
target system.  To ensure library compatibility --- and avoid what is often
called ``DLL hell'' --- a target system may be packaged together with
specific versions of libraries that are known to work with it.  In this
way, developers can ensure that their system will run on any supported
platform regardless of the particular versions of library components that
clients might or might not have already installed.

However, if software components are included without clearly identifying
their origin then a number of technical and ethical concerns may arise.
Technically, it is hard to maintain such a system if its dependencies are
not well documented; for example, if a new version of a library is released
that contains security fixes, system administrators will want to know if
their existing applications are vulnerable.  Ethically, code fragments that
have been copied from other sources, such as open source software, may not
have licences that are compatible with the released system.  When this problem
occurs within proprietary systems, resolution can be costly and
embarrassing to the company.

Many North American financial instutions implement the Payment Card
Industry Data Security Standard~\cite{PCI_DSS_121} (PCI DSS).  Requirement
6 of this standard states: ``All critical systems must have the most
recently released, appropriate software patches to protect against
exploitation and compromise of cardholder data.''  Suppose a Java
application running inside a financial institution is found to contain a
dependency on a Java archive named \mytt{httpclient.jar}.  Ensuring that
the PCI DSS requirement is satisfied entails addressing some difficult
questions:

\begin{compactitem}
\item  Which version of \mytt{httpclient.jar} is the application
    currently running?

\item How hard would it be to upgrade to the latest version of
    \mytt{httpclient.jar}?

\item Has the license of \mytt{httpclient.jar} changed within the newest
    version in a way that prevents upgrading?

\end{compactitem}

We can use a variety of techniques to address these questions.  For
example, if we have access to the source code we can do software clone
detection. If we have access to binaries, we can perform clone analysis of
assembler token streams, call flow graph matching, string matching, mining
software repositories, and historical analyses.

This kind of investigation can be performed at various levels of
granularity, from code chunks to function and class definitions, to files
and subsystems up to compilation units and libraries.  But the fundamental
question we are concerned with is this:  given a software entity, can we
determine where it came from?  That is, how can we establish its
\emph{provenance}?

\section{Contributions}

%We summarize the contributions of this paper as follows:
\begin{enumerate}
\item We introduce the general concept of \emph{software Bertillonage}, a
    method to reduce the search space when trying to locate a software
    entity's origin within a corpus of possibilities.

\vspace{0.7em}
\item We present an example technique of software Bertillonage: anchored
    signature matching.  This method aids in reducing the search space when
    trying to determine the identity and version of a given Java archive
    within a large
    % authoritative     % JWD, Sep 5th, 2011:  I don't consider Maven authoritative!  It's the wild west!
    corpus of archives, such as the Maven 2 central repository.

\vspace{0.7em}
\item We establish the validity of our method with an empirical study
    of 945 binary jars from the Debian 6.0 GNU/Linux distribution.
    We demonstrate the significance of our method by replicating
    a case study
    of a % to find exact version information of binary jars used in a
    real world e-commerce application containing 81 binary jars.

\end{enumerate}

\input chapters/1/110_bertillonage
\input chapters/1/120_background

