
\section{Bertillonage and Software Provenance{\label{sec:bertillonage}}}

In the mid to late 19th century, police forces in Europe and elsewhere
began to take advantage of emerging technologies.  For example,
suspected criminals in Paris were routinely photographed upon arrest, and
the photos were organized by name in a filing system.  Of course, criminals
soon found out that if they gave a false name upon being arrested then
their chances of being identified from the huge pool of photos was very
small unless the police were particularly patient or happened to recognize
them from a previous encounter.  Alphonse Bertillon, the son of a
statistician who worked as a clerk for the Paris police, had the idea that
if suspects could be routinely subjected to a series of simple physical
measurements --- such as height, length of right ear, length of left foot,
etc. --- then the photos could be organized hierarchically using the
bio-metrics data, and the set of photographs that had to be examined for a
given suspect could be reduced to a small handful.
%changed subsequently to later, to gain one line
This approach, later termed \emph{Bertillonage} in his honour, proved to be
very effective and was a huge step forward in the burgeoning science of
criminology~\cite{encyForensic2000}.

As a forensic approach, Bertillonage also had its drawbacks.  Using the
specialized measuring equipment required extensive training and practice to
be reliable, and it was time-consuming to perform.  Each of 10 measurements
was performed three times, because if even one measurement was off then the
system did not work.  Also, the measurements taken did not have a high
degree of independence; tall people tended to have long arms
too.\footnote{The interdependence of the Bertillonage bio-metrics was
recognized by Francis Galton, and it inspired him to devise the notion of
statistical correlation.} In time, the emerging science of fingerprinting
proved to be a much more effective and accurate identification mechanism
and Bertillonage was forgotten.  Nevertheless, Bertillon and his other
inventions --- including the modern mugshot and crime scene photography ---
showed how simple ideas combined intelligently could greatly reduce the
amount of manual effort required in forensic investigations. Despite its
limitations, Bertillonage was considered the best method of identification
for two decades~\cite{forensic2006}.

Our goal in this work is to devise a series of techniques to aid in
determining the \emph{provenance} of software entities.  That is, given a
software entity such as a function definition or an included library, we
would like to be able answer the question:  \emph{Where did this entity
come from?}  Of course, most often the answer will be that the entity in
question was created to fit exactly where it is within the greater design
of the system, but sometimes entities are moved around, designs are
refactored, new is copied from old and then tweaked.  We would like be able
to answer this question authoritatively: this is version 1.3.7 of the X
library; this SCSI driver is a tweaked clone of a driver of a similar card;
most of this function $f$ was split off from function $g$ during a
refactoring effort in the last development cycle, etc.  Sometimes, however,
our answers will be best-effort guesses, especially if we do not have
authoritative access to the original developers.

We therefore use the metaphor of software Bertillonage, rather than, say,
software fingerprinting, as we often lack sufficient evidence to make a
conclusive identification.  Instead, we use a set of simple and sometimes
ad-hoc techniques to narrow the search space down to a level where
more expensive approaches (e.g., a manual determination, or a slow, exhaustive
algorithm) may be feasible.

\section{Our Previous Report}

Compared to the implementation in our previous Software Bertillonage report
\cite{DaviesGGH11}, we have since improved and enhanced our toolset, our
corpus, and our experiments.  We have abandoned the source parser that we wrote
from scratch.  Instead we use Java's own compiler, \mytt{javac}, from
Oracle's 1.6.0\_20 release of Java, to
analyze source code.  We have also switched our bytecode analyzer from
\mytt{bcel-5.2.jar} to \mytt{asm-3.3.1.jar}.  Thanks to these improvements we can
now extract more features from source and binary Java artifacts, such as generics, enums, and
inner classes.  We obtained a new snapshot of the Maven2 repository to
serve as our provenance corpus.  Surprisingly, the Maven2 repository has
nearly doubled in size since our previous report, from 150 GB to 275 GB.
We previously used an industrial case study to explore the feasibility of our
main ideas.  We now test our improved techniques and tools with an
empirical experiment based on 945 jar files of known provenance, as well as
a replication of the original case study.


\section{Replication}
\noindent Data for replication is available at:\\
\url{http://juliusdavies.ca/2013/j.emse/bertillonage/}
