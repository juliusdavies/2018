\section{Related Work{\label{sec:background}}}

In software engineering research, similar questions relating to development
artifact provenance and attribution have been addressed in various guises.
For example, there is a large body of work in software clone detection that
asks the question: which software entities have been copied (and possibly
tweaked) from other software entities.  Our own work~\cite{Godfrey2005tse}
on the problem of ``origin analysis'' asked: if function $f$ is in the new
version of the system but not the old, is it really a new function or was
it merely moved / renamed / merged or split from another entity in the old
version?  The emphasis in our work here is to broaden the question even
further.  Given the recent advances in the field of mining software
repositories, can we take advantage of the vast array of different
software development artifacts to draw conclusions about the provenance of
software entities?

%Julius removed {Krinke2007} to get us down to 10 pages on August 27th, 2010

There exist many studies on the origin, maintenance, and evolution of code
clones~\cite{KapserG08,Krinke2008,Lozano2008,Lozano2007,ST09}, while others
have examined clone lifespan and genealogy~\cite{Kim2005fse}.  The
distinction between these studies and our own is that we study provenance
across applications, and we are interested not only in finding similar
entities, but determining where they come from.  We are also interested in
matching similar entities when one of them is in compiled (binary) form.

Clone detection methods (such as~\cite{KamiyaKI02,LivieriHMI07}), as well
as the tracking of clones between applications~\cite{GermanPGA09} provided
a starting point for our investigation.  Similar to Holmes et
al.~\cite{HolmesWM06} we build our own code-search index.

Di Penta et al.~\cite{DiPentaMSR2010} used code search engines to find the
source code that corresponds to a Java archive (they used the fully
qualified name of the class). They found that their main limitation was the
inability to match a binary jar with the precise version of the source release
it came from.
Similarly, Hemel et al.~\cite{HemelKVD11} showed how extracting
string literals from binaries to detect clones can work surprisingly well,
often out-performing other more sophisticated techniques.
Ossher et al.~\cite{OssherSL11} employ a technique they call
``name-based fingerprints'' in their source-based clone analysis of the Maven 2
Central repository; these fingerprints are a simplified version (e.g., no inner classes,
no return types) of our \emph{anchored class signatures}.
We consider all of these works to be forms of Bertillonage.

Recently, a line of research on software development ``recommender
systems'' has arisen \cite{CubranicMSB05,HolmesW10,KerstenM05,RobillardWZ10}.  The
goal here is to analyze a given working context --- such as the bug report
being worked on, the source code files that have been changed, the API
elements whose documentation has been accessed --- and try to infer what
other artifacts (bug reports, API elements, documentation) might be
relevant to the development task at hand.  This is done using historical
usage information, which can be specific to a developer, a team, or
use a public history repository.  This can be seen as another instance
where is it desirable to characterize software artifacts and perform a
loose matching algorithm on them against a large repository.  The matching
algorithm must be loose to be useful, since it is highly unlikely that the
exact combination of artifacts have ever been used at the same time before.


% Di Penta and German et al.'s continued work on FOSS license usage and
% patterns \cite{dmg2009icse}, and license auditing \cite{icpc2010}
% provides background in current issues surrounding intellectual
% property and open source software engineering.  The recent license

% evolution \cite{icse2010} work in particular provides important
% motivation for our study.


% Maven2 acts like a major repository of Java source code and jars
% archives. They are organized into directories based on the name of the
% package. Within each of these directories one can find source code
% and/or binary archives. Maven2 wants to become a central repository that
% can be used for the dependency management tool Apache
% Maven\footnote{\url{maven.apache.org}}. There is little organization
% and no apparent documentation in terms of what goes into each of these
% directories.  For example, the directory \mytt{myfaces} contains two
% different copies of the same jar: \mytt{myfaces-1.0.9.jar} (one in the
% main directory and another one in the directory \mytt{myfaces-1.0.9};
% on contrast version 1.0.10 is only present in the directory
% \mytt{myfaces-1.0.10}. It appears that, at this point, Maven2 is an
% stage of gathering Java code, before it is properly organized.

% Nonetheless, because it is one the most comprehensive repository of
% Java code, and it is easily replicable, Maven2's un-versioned
% architecture provides an excellent opportunity to mine for version
% history and code evolution.  Many different versions of any given jar
% file are contained therein.  We aim towards using Maven's
% breadth--over 55,000 distinct Java namespaces, a.k.a. `packages'--and
% its depth--over 61,000 distinct jar names providing multi-version
% coverage of many projects--to improve two aspects of Di Penta et al.'s
% work on code-search described in recent work by Di Penta et al.
% \cite{DiPentaMSR2010}. In that paper, they propose a method to
% identify the source code that created a given binary jar file, but
% found that their main limitation was that using a search engine (such
% as Google Code Search) it is not feasible to match the precise version
% of the source code that created the given binary jar. The work
% describe herein attempts to address that issue.

% \DMG{Let us add a table with stats of maven: number of main
%   directories(different packages), total, and median number of jars,
%   total number of source code archives}

