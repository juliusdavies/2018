\startchapter{Conclusion and Future Work}\label{sec:conclusion}
\label{concl}

In this paper, we have discussed the problem of determining the provenance
of a software entity.  That is, given a library, file, function, or even
snippet of code, we would like to be able to determine its origin:  was the
entity designed to fit into the design of the system where it sits, or has
it been borrowed or adapted from another entity elsewhere?  We argued that
determining software entity provenance can be both difficult and expensive,
given that the candidate set may be large, there may be multiple or even no
true matches, and that the entities may have evolved in the mean time.
Consequently, we introduced the general idea of software Bertillonage:
fast, approximate techniques for narrowing a large search space down to a
tractable set of likely suspects.

As an example of software Bertillonage, we introduced \emph{anchored
signature matching}, a method to determine the provenance of source code
contained within Java archives.  We demonstrated the effectiveness of this
simple and approximate technique by means of an empirical experiment
performed on 945 jars from the Debian GNU/Linux distribution, and using a corpus drawn from
the Maven2 Java library repository.  We found that we were able to reliably
retrieve high-quality provenance information of contained binary Java archives
if the product was present in our database derived from Maven2, and in the majority of cases we were
able to identify the correct version.  If a sought product was not present
in Maven, this was usually quickly obvious.  However, if a product was
present we found that identifying the correct version was sometimes tricky,
requiring detailed manual examination.  The use of anchored signature
matching proved to be very effective in eliminating superficially similar
non-matches, providing a small result set of candidates that could be
evaluated in detail.

Being able to determine the provenance of software entities is becoming
increasingly important to software developers, IT managers, and the
companies they work for. Often these stakeholders need this information  in
order to comply with security standards, licensing and other requirements.
Given the wide ranging nature of the problem, the large candidate sets that
must be examined, and the detailed amount of analysis required to verify
matches, we feel that this is only the beginning of software Bertillonage.
We need to design a wide array of techniques to narrow the search space
quickly and accurately, so that we can then perform more expensive analyses
on candidate sets of tractable size.

\section*{Acknowledgements{\label{sec:acknowledgements}}}

We thank Dr. Anton Chuvakin of Security Warrior Consulting
(www.chuvakin.org) for his advice on PCI DSS.


% %  version-aware
% % and license-aware code-search engine.  Unlike existing
% % code-search engines that expect textual queries,
% % ours is queried by uploading a full jar file.
% % We were able to find source version matches
% % to within a minor version or better
% % over 68\% of the time
% % in a sample of 250 random jar files.  By supplementing our Maven index
% % with missing source files we were able to increase version matching to over 90\%
% % in a sample of 35 jar files.

%   In future work our definition of a class-signature could be altered to leave out the namespace
%   (i.e., the `a.b.' in a.b.C from Listing \ref{lst:cSig}), since package is already
%   available as a column in our database.   Clones are known to often have identical
%   file names, but our current index unnecessarily forces clones to also have identical directory
%   sub-trees (i.e., ./a/b/C.java) since Java name-spaces and directory trees are
%   inseparably linked.  For our current study anchoring our clones in such a way helped us avoid
%   the complexity of filtering results:  all detected clones were always positive matches.
%   However, a looser signature definition could help us find more examples of copy-paste reuse.



% The false candidates motivate
% future improvements to our signature extraction.
% We may be able to properly categorize more of the false candidates as low-confidence
% matches by improving our signature extraction.  Future avenues for improvement include:

% \begin{enumerate}

% \item Inner classes are currently ignored.  Two of the false candidates above
% would become low-confidence matches had we extracted signatures
% from inner classes.

% \item Java stores a string literal pool inside each class file.  This
% could be particularly helpful since Java library release procedures often
% include hard-coding the library's version as a string literal\footnote{
% CVS and Subversion offer a keyword expansion feature that automatically
% inserts a version-specific string literal directly into the source code.}.

% \end{enumerate}

% Improvements to the signature extraction would likely result in smaller
% multiple-candidate sets as well, which in turn would lead to more
% single-candidate results.


